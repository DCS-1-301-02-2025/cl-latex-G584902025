\documentclass[a4paper,11pt,dvipdfmx]{ujarticle}
% パッケージ
\usepackage{graphicx}
\usepackage{url}
% レイアウト指定を記述したファイルの読み込み
\input{layout}

% タイトルと氏名を変更せよ.
\title{日本におけるデジタル化の状況}
\author{G58490-2025 村上 瑛士}

\begin{document}

\maketitle %ここにタイトルが入る

% ここから本文
% 節見出し: \section{}
% を使う
\section{デジタル競争力ランキング}

国際経営開発研究所(IDM)の調査\cite{imd}によると,日本のデジタル競争力のランキングは図\ref{fig.競争力}に示すよ

うに,調査対象64カ国中,総合で28位,技術分野で30位となっている。

\begin{figure}[htbp]
    \centering
    \includegraphics[width=0.7\linewidth]{/Users/murakamieiji/Downloads/画像その1.png}
    \caption{デジタル競争力ランキング(64カ国中)}\label{fig.競争力}
\end{figure}








\section{ブロードバンドの整備状況}

OECDによるブロードバンド回線の普及に関する調査\cite{oecd}によると,表\ref{tbl.利用者数}に示すように,日本における100人あたりの光ファイバー回線の加入者数は29.0で,韓国,スウェーデン,ノルウェーに続いて第
4位になっている.

\begin{table}[htbp]
    \centering
    \caption{光ファイバー回線の加入者数(100人あたり)}
    \label{tbl.利用者数}

    \begin{tabular}{|c|c|c|}\hline
        順位 & 国名 & 加入者数 \\
        \hline
        1位 & 韓国 & 38.2 \\
        \hline
        2位 & スウェーデン & 31.9 \\
        \hline
        3位 & ノルウェー & 39.5 \\
        \hline
        4位 & 日本 & 29.5 \\
        \hline
        5位 & アイスランド & 28.8 \\
        \hline
        6位 & スペイン & 27.3 \\
        \hline
        7位 & ポルトガル & 25.1 \\
        \hline
        8位 & ニュージーランド & 23.6 \\
        \hline
        9位 & リトアニア & 22.3 \\
        \hline
        10位 & フランス & 21.2 \\
        \hline
    \end{tabular}
\end{table}


\section{考察}
\begin{itemize}
    \item 日本のデジタル競争力は平均的である。
    \item 光ファイバー回線の加盟者数は上位である
\end{itemize}
日本のデジタル競争力はあまり高くはないが、光ファイバーの普及率は高いため、
技術に対する国民の関心と、技術力の間にギャップがあると考えた。
% 本文(1)
%  参考文献の参照: \cite{}
%  図番号の参照: \ref{}
% を使う
% 文献データベースのキーワードは oecd と imd
% になっている.

% 図の挿入
% \includegraphics{}
% を
% \begin{figure}[htbp]
% \end{figure}
% で囲み
% \caption{}
% で図のタイトルを入れる.
% \label{}
% を使って図番号が参照できるようにする
% また,
% \centering
% で図が中央に来るようにする

% ーーー
% 節見出し(2)

% 本文(2)

% 表の挿入
% \begin{tabular}
% \end{tabular}    
% による表の記述を 
% \begin{table}[htbp]
% \end{table}
% で囲み
% \caption{}
% で表のタイトルを入れる.
% \label{}
% を使って表番号が参照できるようにする
% また,
% \centering
% で表が中央に来るようにする

% ーーー
% 見出し(3)
% 考察
%
% \begin{itemize}
% \end{itemize}
% を使って箇条書きで記述する

% ここに参考文献が入る
%


\bibliographystyle{junsrt}
\bibliography{exercise.bib}

\end{document}